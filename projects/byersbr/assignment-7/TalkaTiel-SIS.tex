\documentclass[12pt]{article}
\usepackage[margin=1in]{geometry}
\usepackage{graphicx}
\graphicspath{ {./} }


\title{Talkatiel Software Requirements and Planning}
\author{Brendan Byers, Ryan Sisco, Iliana J, Aidan Grimshaw}
\date{\today}

\begin{document}
\begin{center}
 \Large\textbf{TalkaTiel Second Implentation System}\\
 \large\textit{Brendan Byers, Ryan Sisco, Iliana J, Aidan Grimshaw, Yufei Zeng}\\
 \large{byersbr, siscor, javieri, grimshaa, zengyu}\\
\end{center}

\tableofcontents
\section{Product Release} \subsection{Server Side Implementation} Currently the
server code can be downloaded and run locally.  By cloning the repository
located at\begin{verbatim} https://github.com/B13rg/Talkatiel_API.git
\end{verbatim}one can run the server.  There is also extensive documentation
detailing all the steps that need to be taken to run the server.  There is also
a SQL script that will create a database when run.  This can be used to create a
database for use on a different engine.  Currently it is configured to run on
the localhost on port 5002.  This can be tested to navigating to
\begin{verbatim}127.0.0.1:5002/Posts/New \end{verbatim}where you can see the raw
Json output of a post.  In the repository there is also an sqlite database that
is run alongside the python code.  This will connect with the python to respond
to different sql queries.  It is fully functioning, and the only thing left to
do is test it, and find a permanent URL. Additionally, you can test GET and POST
reqeusts at our API server URL, aidangrimshaw.pythonanywhere.com, using the
documentation located on our API github repo.

\subsection{Client-Side Implementation}
By midweek, we plan to start permanent
hosting with google app engine.
Product URL:\begin{verbatim} https://github.com/thegrims/talkatiel-ui \end{verbatim}
The product is current working on a basic level. We are able to consistently pull data and connect to the server. We have added several test posts and are able to bring them into cards on our mobile client. We are currently working on developing our fingerprinting and identifying users individually and tying them to their posts server side only. This will take some time and testing, and will only be implemented after a series of testing and penetration testing. When we implement this, we need to be absolutely sure that the data is secure, because we will be dealing with sensitive data. 


Our CSS is looking as intended. We have built it to be easy to change later on. We are not completely decided on a color, however it works with most. We have yet to implement animations. We do have screen rotation completed for mobile devices, and the app grows in size accordingly. We have tested the app on IOS and Android, with a consistent look between the two.
\section{User Story}
\subsection{Add a post}

\subsection{Add a Comment}

\subsection{Like/Dislike}

\subsection{Report}

\subsection{Delete}

\subsection{Refresh}

\subsection{Sort}

\subsection{View Newly Created Post}

\subsection{Secure HTTPS Connection}

\subsection{Save Website as App on Phone}

\subsection{Responsive Web Page}

\section{Tests}
\subsection{Server Side Implementation}

\section{Design Changes and Rationale}
\subsection{Server Side Implementation}
For the most part the server group stayed very close to original design.  To
design the database, we worked off the UML design diagrams we designed earlier.
This allowed us to see what fields we needed to include for each table.  We
added an extra table not originally in the design doc to handle reports.  We
found there was no way to keep track of reports server side, so we added a table
to handle it.  It keeps track of the user who reported it, the post in question,
and the text of the user’s report.  By having its own table, it can also be
queried separately instead of having to sort through who knows what.
Additionally, we modified our api server implementation so that it was served
remotely from a different service than the client side server, and so that the
API would be better equiped to handle POST requests.
\subsection{Frontend Implementation}
We have stayed very close to our original design. We currently are finishing up
our implementation of the main posts page, and will switch our focus to the
implemetation of the posts page and further code cleanup / refactoring of the
posts page. Visual elements are styled slightly differently for ease of use, and
commenting functionality is removed in this version to ease the implementation
of our minimum viable product (MVP).

\section{Refactoring}

Our project is still being developed in parts. For the most part, we have
been working towards building a working application and database, and slowly
cleaning up as we go. This has lead to some sloppy, duplicate code. However,
we have been making progress building. Once one of our members has completed
his/her section, while waiting for others to complete, they have cleaned up
their code and make it easier to follow. While this isn’t a top priority
right now, we believe that refactoring our code as we complete files will
help us in the future. It will also allow us to switch duties and work on
each other's code without too much confusion.

We ended up doing minor refactoring. While working on the client side
development, we ended up making templates for common, reoccurring elements.
This has been useful while developing other parts of the application, when
attempting to make the styling similar. We have been able to remain somewhat
consistent.

In addition to these changes, we have also changed variable names before
committing to the shared repository. This helps those who are viewing
changes follow along. Many of us have been saving files offline before
pushing, forcing us to correct changes before pushing. By doing this, we are
not stressed to push our work as a simple “backup” of our work. Saving and
updating locally/remotely before pushing to a shared repo has helped us stay
consistent and organized.

Making these changes has helped us stay organized and also limit the
“turnover” time between switching parts and tasks. For instance, when two
people switched files to review code, we were able to follow along much
easier when the code was simplified.

\section{Tests}



\section{Meeting Report}
\subsection{Schedule for next week}

\subsection{Progress made this week}

\subsection{Plans and goals for next week}
\subsection{Team Member Contributions}

\end{document}
