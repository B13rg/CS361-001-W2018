\documentclass[12pt]{article}
\usepackage[margin=1in]{geometry}

\title{Talkatiel: Anonymous Local Bulletin Board}
\author{Brendan Byers}
\date{\today}

\begin{document}
\begin{center}
      \Large\textbf{Talkatiel: Anonymous Local Bulletin Board}\\
      \large\textit{Brendan Byers}
   \end{center}

Social media is allowing people to fence themselves off, creating an echo chamber with the potential to spread misinformation.  Facebook and Twitter allow a user to tuen their profile so much that one is able to surround themselves only with opinions that agree with their own, never even taking into account other viewpoints.  So much bias or blatantly false information is passed around, causing people to fold into their own beliefs and ignore any outside influences.  By allowing freer communication, people are able to expand their viewpoints and develop a better world view.

I propose to create a location based, chatting portable web app (PWA) that would allow people to anonymously chat and discuss anything.  This has been done with limited success in the past with apps like Yik-Yak, but a few issues caused it to eventually be rejected by users.  It began as an anonymous, location based chat app.  A big draw to the app was the anonymity.  It allowed you as a user to say whatever you wanted without having to worry about repercussions.  It also allowed you to see how people actually think and behave in your area without the pressure to be socially and politically correct.  Sadly, beginning in March of 2016 Yik-Yak began to modify the way the app behaved.  They rolled out usernames and profiles, implementing them over time in an attempt to convert their current audience to the new way of doing things.  As it developed, it removed many of the features that made the app really great.  You could no longer posts anonymously, and the app focused around developing your “profile” instead of focusing on original discussions.  

They built their entire app around anonymous, location based chats but misinterpreted what their target audience really wanted.  The app turned into a way for people to meet and hangout in real life, which is definitely not what people wanted.  They’ve ostracized their user base by removing anonymity from the app, and attempting to guide users into meeting up with each other.  This was never the original intention of the app, and is very different from what users of the app really want.  

Seperate from the development and “improvements” added to Yik-Yak, another issue a lot of users had with the app with moderation ability.  Often people would post things that were offensive or examples of cyberbullying.  There were no true moderation tools in place to stop potentially harmful posts from being viewed.  Individual people would be singled out, creating an unhealthy environment with no repercussions for the perpetrator.  This alienated many users and caused them to leave the platform.  While to structure was there for users, it often did not do enough to let users feel welcome.

I want to make an PWA that achieves the goals of the original Yik-Yak formula while fixing some of the shortalls present in the original system.  It would be an app that has an emphasis on anonymity and fostering discussion.  It would be developed as a PWA because of how lightweight and easy to use they are.  It doesn’t require users to download an app, meaning that there is one less barrier to entry.  It would take into account location, and allow users to participate in different “flocks” or discussion rooms.  People would be able to create threads with in discussion rooms that people can respond to.  Posts will be able to be positively or negatively graded which effects their ranking.

Flocks will be directly tied to locations.  Each flock will have a single location where it’s “center” is.  Flocks will have different preferences that affect who can view them.  These preferences will include: location, range, privacy settings and theming.  Location and range will be closely tied together.  A flock will be tied to a single location, and will only allow people within a certain range from that location to view and participate.  The flock range will be able to be as small as 50-100 and have no upper limit, allow the flock to be accessed from anywhere.  As for privacy settings, the creator or moderator of the flock will be able to define many settings that allow the flock to be tuned to its intended users.  Moderators can adjust different filters to allow or not allow people to join a flock.  This could include things such as requiring a password to access a flock, only allowing certain types of users, or viewing/posting requirements.  Finally, the moderators will be able to “theme” their flocks, allowing them to adjust flock color themes and other UI elements. 

An ever-present issue with having a project such as this is enforcing and controlling speech rules while also allowing users to speak freely.  There will always be people who act in a way that is not wanted on the platform.  To try and stop users from bullying and abusing others, strong and encompassing moderation tools must be created in order to equip flocks with the tools to stop potentially harmful posters.  There will be things in place to filter out or “shadowban” users, making it appear to them that they are posting but in reality their posts never appear for everyone else.  Another popular site, Reddit, also deals with issues moderating the free speech of users.  Following their design can allow this project to enforce quality control while also allowing people to speak about what they want, to a degree.  There will be ways to positively identify users by their posts if need be.  Yik-Yak ran into trouble with lawsuits concerning the use of racist, sexist, aggressive and threatening language because it was unable to identify users.  There will be things in place that allows the group behind Talkatiel to enable that privilege.  This means that users of the app won’t be able to participate through proxies, VPN’s, or other privacy-preserving resources.  Talkatiel will try to promote a friendly and healthy environment while ensuring that malicious users are unable to ruin the experience for others.

The PWA will make use of HTML, CSS, Javascript, and SQL.  The PWA itself will have a simple frontend with a database in the backend.  The PWA will be delivered and cached on a users device, meaning faster lead times and an improved user experience.  The database backend will contain all the content created and deliver it users as they request it.  IndexedDB is a noSQL database that stores data as key-value pairs in object stores.  A version of the master DB will be replicated onto each users device, creating a faster application.  By caching things we can also lessen the load on our own database, meaning less latency and better service overall.

I believe that if done correctly, this project could lead to a successful app that would fill the space left by Yik-Yak.  By focusing on and refining what the original idea did right, Talkatiel has the potential to be just as successful while enabling users to expand their viewpoints and be informed of the beliefs of others.  Communication is one of the foundations that modern society is built on.  Social media has snuck its way into our lives and created miniature worlds that we all live in.  The private worlds surround us people of similar thoughts and opinions, and restrict us from being able to truly experiance and understand differeing or contesting beliefs.  Talkatiel enables users to expand their own horizons and talk with people they may never encounter in their dail lives.

\end{document}
